\chapter{Machine Implementation}
\label{chp:machImplmnt}

\section{Machine}
The implementation details for the machine are given below. details given include the header file, the corresponding code file, the various functions included in them and a short description of these functions.

\begin{enumerate}
	\item \textit{data.h }
	
	Consists of constants declared for the machine. These include the registers, and the size of various constiuents of memory. The entire memory is declared here as well.
	
	\item \textit{memoryConstants.h}
	
	Declarations for the structure of main memory is made here.
	
	\item \textit{instr.h}
	
	Declares the constants associated with each token.
	
	\item \textit{decode.lex}
	
	This is the lexical analyser which analyses each instruction and reurns the corresponding token.
	
	\item \textit{boot.h} and \textit{boot.c}
	
	This file consists of the following functions:
	\begin{itemize}
		\item void loadStartupCode() - Loads the OS Startup Code from disk to the proper location in memory
		\item void initializeRegs() - Initializes the values of all the registers to zero.
	\end{itemize}
	
	\item \textit{scheduler.h} and \textit{scheduler.c}
	
	This file consists of the following functions:
	\begin{itemize}
		\item void runInt0Code() - Causes the timer interupt leading to the control being passed on to the INT 0 code in memory.
	\end{itemize}

	\item \textit{timer.h}
	
	This file contains the constant defining the number of clock cycles that makes up a timslice alloted to a single process. This file also consists of the following functions:
	\begin{itemize}
		\item int isTimeZero() - Checks whether the timer counter reads zero.
		\item void tick() - Decrements the timer counter.
		\item void resetTimer() - Resets the timer counter.
	\end{itemize}
	
	\item \textit{utility.h} and \textit{utility.c}
	
	This file consists of the following functions:
	\begin{itemize}
		\item void emptyPage(int) - Clears the page speecified by the aguement.
		\item struct address translate(int) - Translates the virtual address passed as arguement to the corresponding page number and offset.
		\item printRegisters() - Prints the values of all the registers. Used for debugging purposes.
		\item void exception(char*) - Acts as the exception handler. Prints the instruction that caused the exception and terminates execution.
	\end{itemize}

	\item \textit{simulator.h} and \textit{simulator.c}

	This file consists of the following functions:
	\begin{itemize}
		\item void Executeoneinstr(int) - This function simulates all the instructions available on the esim architecture. 
		\item void run(int, int) - This function acts as the bootloader. It loads the Startup code. It also calls Executeoneinstr() for every instruction that it reads.
		\item int main(int, char**) - Makes the initial changes to the machine environment and the calls run().
	\end{itemize}

\end{enumerate}