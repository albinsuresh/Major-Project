\chapter{OS Startup}
\label{chp:osstartup}

\section{OS Startup Code Specification}
\index{OS Startup Code}
\label{lbl:oscode}
When the machine boots up, the \textit{Bootloader} code loads the \textit{OS startup code} into the main memory. The OS startup code (instructions in page 1, see fig~\ref{fig:main memory}) starts execution in the \textit{Kernel mode}. It performs the following functions.
\begin{itemize}
	\item It loads the Interrupt Service Routines from the blocks 1--8 of the hard disk into pages 56--63 of the memory.
	\item It loads the FAT from blocks 11 and 12 of the hard disk into pages 4 and 5 of the memory.
	\item It loads the disk free list from Blocks 9 and 10 into pages 6 and 7 of the memory.
	\item It generates the memory free list and stores it in words 48--111 of page 2 of the memory.
	\item It loads the INIT process from the hard disk into the memory by performing the following steps:
	\begin{itemize}
		\item Load the INIT process from blocks 14--16 of the hard disk to pages 8--10 of memory. Page 11 is allocated as the user stack.
		\item Update the memory free list.
		\item Update the ready list and PID register.
		\item Set the required page table entries.
		\item Set the values of SP, BP and IP with values 768, 768 and 0 respectively.
	\end{itemize}
	\item Switch from \textit{Kernel mode} to \textit{User mode}.\footnote{This can be achieved by calling IRET.}
\end{itemize}

Note: All addresses are absolute addresses in Kernel mode. 
\section{INIT Process}
\index{INIT Process}
\label{lbl:INITprocess}
The Operating System currently supports execution of only a single user program - the INIT process. Testing of the OS startup code can be done by loading the required user program as the INIT process. Modification to INIT will be done later.
